\chapter{Technical Work}
\label{ch:tw}

%In this chapter we will discuss the experiments and the process that has been followed to develop the model for predicting the amplifier output profile


\section{Introduction} 
\label{tw:intro}

This chapter will discuss the experiments that have been carried out and the process that has been followed leading to the development of a machine learning model capable of predicting the output characteristic of an EDFA. 
The work that has been undertaken will be split into sections as follows:
\begin{itemize}
    \item Amplifier characterisation, a detailing of the experiments carried out in order to characterise the amplifier model that has been used throughout this thesis.
    
    \item Data generation will discuss the experiment that has been designed to generate suitable training data for the development of the machine learning model.
    
    \item Model Design will outline the development of the baseline machine learning model, the development environment and the design choices made during the development process.
    
    \item Model Refinement 1 – Sensitivity Analysis, will discuss the analysis of the model from section \ref{model_design}. The aim of this analysis is to determine a strategy for reducing the amount of training data required by the ML model.
    
    \item Model Refinement 2 – Pruning and Quantization, will discuss the optimisation methods that have been applied to improve the model in terms of memory requirements, training time and design efficiency. 
\end{itemize}


%%%%%%%%%%%%%%%%%%%%%%%%%%%%%%%%%%
% Beginning of New Section Here
%%%%%%%%%%%%%%%%%%%%%%%%%%%%%%%%%%

\newpage
\section{Amplifier Characterisation}
\label{tw:sec:amp_char}


\subsection{Introduction}
This section will detail the characterisation experiment that was carried out on the amplifier used throughout this project. In place of a physical EDFA amplifier a virtual model from a photonics simulation software package was used. A detailed description of the software package used will also be provided in the following section.	

The need to carry out an experimental characterisation of the EDFA arises from the necessity to verify the accuracy of the virtual model to one of a physical amplifier. This process will determine characteristics of the amplifier, such as the gain profile, the impact of noise on the amplifier output and the amplifier fidelity, in terms of the presence of saturation effects and interaction effects between the various input channels. The results of this characterisation will be used to inform the design of later experiments, and the development of the machine learning model itself.

\subsection{Amplifier model}
An EDFA model from the VPI Photonics software suite (VPI) was used for this experiment. VPI is a photonics design tool, which was specifically suited to the modelling of fibre optics and optical amplifiers. There are a variety of amplifier models available in from the VPI software. 

A black box amplifier, “AmpBlackBoxOpt” will be used throughout this report. It is a model of an EDFA amplifier based on data measured from a physical amplifier. The amplifier output accounts for wavelength-dependent gain and noise spectra. The amplifier operation can be controlled by a set gain value, a set output power value, or can operate in constant saturation mode.

This experiment is to characterise the amplifier model, this is necessary in order to verify that the amplifier will operate as expected under a simulated network traffic load.  The experiment should:

\begin{itemize}    
    \item Determine the output gain profile. This will show whether a gain flattening filter has been applied to the amplifier output.\\

    \item Determine if the amplifier model has a saturation point, and if so what it is.\\
    
    \item Verify that the amplifier model accounts for interaction effects between input channels.\\
\end{itemize}





\subsection{Experimental Setup}
The experimental setup to characterise the amplifier is shown in figure \ref{fig:tw_amp_char}.
The amplifier input consists of 3 optical signal sources multiplexed together:

\begin{figure}
    \centering
    \includegraphics[width=\linewidth]{images/technical_work/section_1_characterisation/amp_char_ex_setup.png}
    \caption{Experimental Setup in VPI}
    \label{fig:tw_amp_char}
\end{figure}


\begin{enumerate}
    \item The primary input source, the generated by a WDM comb module outputting 44 channels spaced in 100GHz intervals, over the range: 192.05THz - 196.45THz. This frequency range and spacing was chosen as it matched the frequency spacing of the ITU DWDM frequency grid. The output channel power was set at 50$\mu$W as to emulate a realistic input to an EDFA that might occur in an actual network scenario. \\
    
    \item The lower variable source, this optical source will output a single channel at a frequency of 192.535THz, the output power of the signal will be varied during the course of the experiment.\\
    
    \item The upper variable source will mirror the lower variable source at a frequency of 195.935THz and will also be varied during the course of the experiment.  
\end{enumerate}


The three signal sources are combined using an ideal multiplexer and inputted to the black box amplifier. The output of the amplifier was sent to an optical signal analyser to be graphed.

The output powers of the lower and upper sources were varied from a value of 1$\mu$W to 1W in 5 steps. Only optical signal from either source was active during a given step.

\subsection{Results}


The results of the experiment can be seen in figures \ref{fig:tw_amp_char_lc} and \ref{fig:tw_amp_char_uc} , where the output spectrum as a result of varying the lower source is shown in figure and as a result of varying the upper source is shown in figure. 

From the figures it can be clearly seen that the output spectrum of the amplifier is non-flat. By examining the change in the output spectrum as the two sources are varied the interaction effects between channels can be seen. The higher frequency channel has a much greater impact on the overall output spectrum than the lower frequency channel of the same power. The gain of the entire output spectrum is suppressed as the power of the high frequency channel is increased. Whereas increasing the power of the lower frequency channel results in a marginal increase in the output spectrum. 



\renewcommand{\arraystretch}{0.5}
\begin{figure}
    \floatpagestyle{empty}
    \centering
    \caption{Shows the effect the power of the lower channel has on the gain spectrum.}
    \begin{tabular}{c}
        \includegraphics[width=0.8\linewidth]{images/technical_work/section_1_characterisation/steps/1.png} \\
        \includegraphics[width=0.8\linewidth]{images/technical_work/section_1_characterisation/steps/3.png} \\
        \includegraphics[width=0.8\linewidth]{images/technical_work/section_1_characterisation/steps/5.png} \\
        \includegraphics[width=0.8\linewidth]{images/technical_work/section_1_characterisation/steps/7.png} \\
        \includegraphics[width=0.8\linewidth]{images/technical_work/section_1_characterisation/steps/9.png} \\ 
    \end{tabular}
    \label{fig:tw_amp_char_lc}
\end{figure}

\begin{figure}
    \floatpagestyle{empty}
    \caption{Shows the effect the power of the upper channel has on the gain spectrum.}
    \centering
    \begin{tabular}{c}
        \includegraphics[width=0.85\linewidth]{images/technical_work/section_1_characterisation/steps/2.png} \\
        \includegraphics[width=0.85\linewidth]{images/technical_work/section_1_characterisation/steps/4.png} \\
        \includegraphics[width=0.85\linewidth]{images/technical_work/section_1_characterisation/steps/6.png} \\
        \includegraphics[width=0.85\linewidth]{images/technical_work/section_1_characterisation/steps/8.png} \\
        \includegraphics[width=0.85\linewidth]{images/technical_work/section_1_characterisation/steps/10.png} \\ 
    \end{tabular}
    \label{fig:tw_amp_char_uc}
\end{figure}



\FloatBarrier
\subsection{Conclusion}


%%%%%%%%%%%%%%%%%%%%%%%%%%%%%%%%%%
% Beginning of New Section Here
%%%%%%%%%%%%%%%%%%%%%%%%%%%%%%%%%%


\newpage
\section{Data Generation Experiment}
\label{tw:sec:data_gen}

\subsection{Introduction}

In this section  the experiment used to generate training data for the ML model will be discussed. The training data must emulate optical signal traffic from real world scenarios to provide the model with sufficient data to be able to predict the outputs of actual amplifiers used in optical networks.

The ITU define a frequency grid for DWDM applications centred at 193.1 THz and using channel spacings up to 100GHz. This frequency range from 1530nm (195.943THz) to, 1565 (191.561THz), known as the C band in fibre optic communications. The C band is the primary wavelength used for long distance optical communication, as it is in this frequency band that fibre optic cables have lowest losses. 

The channel spacing was set at 100GHz to conform to the standard ITU DWDM frequency grid. A common channel plan found in many DWDM systems is to use 44 channels spaced at 100GHz or 88 channels spaced at 50GHz. In this case 44 channels will be used.  

The channel power was set at 50$\mu$W, as this is reflective of a typical channel input power for an EDFA in an optical communication network. The channel power was not varied between channels.

Maintaining a constant channel power has several implications for the model that will be trained on this data. As a result of applying this restriction the model will not account for cases where the power of all channels is varied form 50[mu]W, or where the channel powers are not uniform. 

This restriction will reduce variability in the training data and enable the model to learn the effects of different channel loading conditions more easily. The impact that channels have on each other will be solely due to their relative frequency, as opposed to their relative power, and relative frequency.


Each training data sample will consist of two arrays, both 44 units long. The array of input powers randomly generated by the VPI simulator and, the corresponding array of output powers measured from the output of the amplifier. The training samples can vary in the following ways:
\begin{itemize}
    \item The input power of each of the 44 channels will either be 0W or 50$\mu$W. The number and location of the nonzero channels will be randomly set.
    
    \item The output powers will either be, 0$\mu$W if the input power of that channel was 0$\mu$W, or will be some non-zero value of power determined by the channel specific gain.
\end{itemize}

The gain of each channel will depend on the channel loading conditions, i.e., which channels are on and the relation between a given channel and all other “on” channels.




\subsection{Experimental Setup}
\FloatBarrier

\begin{figure}
    \caption{Experimental setup in VPI. Figure \ref{fig:tw:top_lvl} shows the top level simulation. Figure \ref{fig:tw:co_sim_galaxy} shows the expanded co-simulation sub-module, referred to as a galaxy.}
    \begin{subfigure}{\textwidth}
        \centering
        \caption{From left to right, the co-simulation galaxy, EDFA amplifier and the optical systems analyser.}
        \includegraphics[width=0.8\textwidth]{images/technical_work/section_2_data generation/ex_setup.png}
        \label{fig:tw:top_lvl}
    \end{subfigure}
    \begin{subfigure}{\textwidth}
        \centering
        \caption{Expanded co-simulation galaxy, showing the co-simulation model being used as an output.}
        \includegraphics[width=0.8\textwidth]{images/technical_work/section_2_data generation/co_sim_galaxy.png} 
        \label{fig:tw:co_sim_galaxy}
    \end{subfigure}
    \label{fig:tw:exp_setup}
\end{figure}


The experimental setup to generate the training data is shown in figure \ref{fig:tw:exp_setup}. The experiment will generate a random channel loading scenario. A random number of channels $n$, between 2 and 44 inclusive, will be turned “on” i.e., have an input power of 50[mu]W. The frequency of each “on” channel will also be randomly decided. The scenario where the input is a single channel will be ignored. As in this case the single channel will not be impacted by the presence of any other channels.

The $n$ channels will be inputted to the amplifier. The value of the corresponding $n$ output channels along with the $n$ input channels will be recorded. This process will be repeated several thousand times to generate sufficient data to train the ML model. 

As in the experiment from section \ref{tw:sec:amp_char}, the amplifier model from VPI will be used. However, there is no straightforward method to generate a random array of channels in VPI. As such the array of input channels will be generated using MATLAB and then be read into VPI using the simulators co-simulation function.

As seen in figure \ref{fig:tw:exp_setup} the experimental setup consists of setting the output of the MATLAB co-simulation module as the input to the optical amplifier, and measuring the output of the optical amplifier using the optical systems analyser module. 


\subsubsection{MATLAB Channel Generation}

Though the optical channels are generated in MATLAB, for them to be read by VPI they must conform to the internal structure that VPI has defined for optical signals. 

Optical signals in VPI are comprised of noise bins, sampled bands, and parameterised signals, all three signal components must be created for a signal to be used in VPI. 
For the purposes of this experiment, we are only interested in the parameterised signals.

The VPI optical signal components are represented by cell arrays in MATLAB, a cell array for each component will be created, but only the parameterised signal cell array will contain the values of our channels.  

The MATLAB code used to generate the channels can be found in appendix [ref].

The MATLAB script is run each time a training sample is generated. Before a sample is generated the random number generator used to select the channels is re-initialised using the current time. This ensure that the channels generated are indeed random and independent from each other. 



\subsubsection{Verification}

A verification script was created to ensure that the channel generation scrip was functioning as expected.

The verification script records the output of the channel generator over a number of runs and plots the results. The results of performing 5000 runs can be seen in figures \ref{fig:tw:data_gen:rel_ch} and \ref{fig:tw:data_gen:rel_num_ch}. 

As can be seen from the figures there is no bias towards any particular channel nor towards a particular number of channels. The variation in the relative occurrence of the channels and the number of channels generated can be attributed to the number of dummy runs that were produced and is expected to decrease were the number of runs increased. Ensuring that there are no biases in the underlying data used to train the model is critical. Further discussion on this are can be found in section [ref]. The script used to verify the channel generator can be found in appendix [ref].
\begin{figure}
    \centering
    \caption{Results of 5000 dummy runs of the channel generator.}
    \label{fig:tw:dummy_runs}
    \begin{subfigure}{0.5\textwidth}
        \centering
    \caption{Plot of the relative occurrence of each channel.}
    \includegraphics[width=\textwidth]{images/technical_work/section_2_data generation/rel_occur_ch.png}
    \label{fig:tw:data_gen:rel_ch}    
    \end{subfigure}
    \begin{subfigure}{0.5\textwidth}
        \centering
        \caption{Plot of the relative occurrence of the number of channels generated, note the y-axis i scaled by $1e^{-3}$.}
        \includegraphics[width=\textwidth]{images/technical_work/section_2_data generation/rel_occur_num_ch.png}
        \label{fig:tw:data_gen:rel_num_ch}
    \end{subfigure}
\end{figure}


\FloatBarrier
\subsection{Results}


