\chapter{Technical Work}
\label{ch:tw}

%In this chapter we will discuss the experiments and the process that has been followed to develop the model for predicting the amplifier output profile


\section{Introduction} 
\label{tw:intro}

This chapter will discuss the experiments that have been carried out and the process that has been followed leading to the development of a machine learning model capable of predicting the output characteristic of an EDFA. 
The work that has been undertaken will be split into sections as follows:
\begin{itemize}
    \item Amplifier characterisation, a detailing of the experiments carried out in order to characterise the amplifier model that has been used throughout this thesis.
    
    \item Data generation will discuss the experiment that has been designed to generate suitable training data for the development of the machine learning model.
    
    \item Model Design will outline the development of the baseline machine learning model, the development environment and the design choices made during the development process.
    
    \item Model Refinement 1 – Sensitivity Analysis, will discuss the analysis of the model from section \ref{model_design}. The aim of this analysis is to determine a strategy for reducing the amount of training data required by the ML model.
    
    \item Model Refinement 2 – Pruning and Quantization, will discuss the optimisation methods that have been applied to improve the model in terms of memory requirements, training time and design efficiency. 
\end{itemize}


\newpage
\section{Amplifier Characterisation}
\label{tw:amp_char}


\subsection{Introduction}
This section will detail the characterisation experiments that were carried out on the amplifier used throughout this project. In place of a physical EDFA amplifier a virtual model from a photonics simulation software package was used. A detailed description of the software package used will also be provided in the following section.	

The need to carry out an experimental characterisation of the EDFA arises from the necessity to verify the accuracy of the virtual model to one of a physical amplifier. This process will determine characteristics of the amplifier, such as the gain profile, the impact of noise on the amplifier output and the amplifier fidelity, in terms of the presence of saturation effects and interaction effects between the various input channels. The results of this characterisation will be used to inform the design of later experiments, and the development of the machine learning model itself.

\subsection{Amplifier model}
An EDFA model from the VPI Photonics software suite (VPI) was used for this experiment. VPI is a photonics design tool, which was specifically suited to the modelling of fibre optics and optical amplifiers. There are a variety of amplifier models available in from the VPI software. 

A black box amplifier, “AmpBlackBoxOpt” will be used throughout this report. It is a model of an EDFA amplifier based on data measured from a physical amplifier. The amplifier output accounts for wavelength-dependent gain and noise spectra. The amplifier operation can be controlled by a set gain value, a set output power value, or can operate in constant saturation mode.

This experiment is to characterise the amplifier model, this is necessary in order to verify that the amplifier will operate as expected under a simulated network traffic load.  The experiment should:

\begin{itemize}    
    \item Determine the output gain profile. This will show whether a gain flattening filter has been applied to the amplifier output.\\

    \item Determine if the amplifier model has a saturation point, and if so what it is.\\
    
    \item Verify that the amplifier model accounts for interaction effects between input channels.\\
\end{itemize}





\subsection{Experimental Setup}
The experimental setup to characterise the amplifier is shown in figure \ref{fig:tw_amp_char}.
The amplifier input consists of 3 optical signal sources multiplexed together:

\begin{figure}
    \centering
    \includegraphics[width=\linewidth]{images/technical_work/section_1_characterisation/amp_char_ex_setup.png}
    \caption{Experimental Setup in VPI}
    \label{fig:tw_amp_char}
\end{figure}


\begin{enumerate}
    \item The primary input source, the generated by a WDM comb module outputting 44 channels spaced in 100GHz intervals, over the range: 192.05THz - 196.45THz. This frequency range and spacing was chosen as it matched the frequency spacing of the ITU DWDM frequency grid. The output channel power was set at 50$\mu$W as to emulate a realistic input to an EDFA that might occur in an actual network scenario. \\
    
    \item The lower variable source, this optical source will output a single channel at a frequency of 192.535THz, the output power of the signal will be varied during the course of the experiment.\\
    
    \item The upper variable source will mirror the lower variable source at a frequency of 195.935THz and will also be varied during the course of the experiment.  
\end{enumerate}


The three signal sources are combined using an ideal multiplexer and inputted to the black box amplifier. The output of the amplifier was sent to an optical signal analyser to be graphed.

The output powers of the lower and upper sources were varied from a value of 1$\mu$W to 1W in 5 steps. Only optical signal from either source was active during a given step.

\subsection{Results}
The results of the experiment can be seen in figure [ref]
[insert figure here]

As can be seen from figure [refence] the output spectrum of the amplifier is non-flat. The shape of the output spectrum depends on not only the location but also the power of the additional channel. The additional channel has a significant impact on the other channels once its power was greater than the channels closest to it. The additional channel also had a much greater impact on the entire gain spectrum when it was located at a higher frequency and the higher frequency channels are impacted to a much greater extent than the lower frequency channels.

[need to add conclusion section here]


%%%%%%%%%%%%%%%%%%%%%%%%%%%%%%%%%%
% Beginning of New Section Here
%%%%%%%%%%%%%%%%%%%%%%%%%%%%%%%%%%


\newpage
\section{Data Generation Experiment}
\label{tw:data_gen}

\subsection{Introduction}

In this section  the experiment used to generate training data for the ML model will be discussed. The training data must emulate optical signal traffic from real world scenarios in order for the ML model to be applicable to actual amplifiers used in optical networks.

The ITU define a frequency grid for DWDM applications centred at 193.1 THz and using channel spacings up to 100GHz. This frequency range from 1530nm (195.943THz) to, 1565 (191.561THz), known as the C band in fibre optic communications, is the primary wavelength used for long distance optical communication. The lowest losses optical fibre losses occur in this frequency band. The experiment must generate data that conforms to the standard ITU DWDM frequency grid. 

Initially bounds for the experiment must be set, these include the number of channels to be used, the channel spacing, the channel power and the variation of channel power.
The channel spacing was set at 100GHz in order to fit with the ITU grid. A common channel plan found in many DWDM systems is to use 44 channels spaced at 100GHz or 88 channels spaced at 50GHz. In this case 44 channels will be used.  Finally, the channel power and power variation between the channels must be set. The channel power was set at 50$\mu$W, a typical input power for EDFAs in use in optical communication systems. The channel power is constant for all input channels. Using a constant channel power restricts the (the model) somewhat, as a result the total input power to the amplifier will also be constant and there will not be any variation between channel powers. 

Applying this restriction will mean that the training data can be varied in the following ways

\begin{itemize}
    \item The input power of each of the 44 channels will either be 0W or 50$\mu$W\\
    
    \item The output power of each of the corresponding channels will either be 0W, if the input power was 0W, or a non-zero value of power determined by the channel specific gain.\\
\end{itemize}


The gain of each channel will depend on the channel loading conditions, i.e., which channels are on or off for that particular sample.



\subsection{Experimental Setup}


The experiment will generate a random channel loading scenario, a random number of channels, between 2 and 44 inclusive, will be set to “on” i.e., have an input power of 50$\mu$W. The location of each “on” channel will also be randomly decided. 
Theses n channels will be the inputs to the amplifier, the inputs and outputs from the amplifier will be recorded to create one training sample. Several thousand samples will be generated to be used to train the model. 

As in the experiment from section [reference prep section], the amplifier model from VPI will be used. However, with the VPI simulator there is no straightforward method to generate a random array of channels that meet the requirements of the experiment. 
As such the input channels will be generated using MATLAB and will then be read into VPI using the simulators Co-Simulate functionality. 


\subsection{MATLAB Channel Generation}

Although the optical channels are being generated in MATLAB in order for them to be read by VPI they must conform to the internal structure that VPI has defined for optical signals. 
Optical signals in VPI are comprised of noise bins, sampled bands and parameterised signals, all three signal components must be created for a signal to be used in VPI. 
For the purposes of this experiment, we are only interested in the parameterised signals.

The VPI optical signal components are represented by cell arrays in MATLAB, a cell array for each component will be created, but only the parameterised signal cell array will contain the values of our channels.  

Each time a training sample is generated the MATLAB script is run. Before a sample is generated the random number generator used in the generating the channels is re-initialised using the current time. This ensure that the channels generated are indeed random and independent from each other. 


\subsection{Verification}	
A verification script was created in order to ensure that the channel generation scrip was functioning as expected.

The verification script records the output of the channel generation scrip over a set number of runs and plots the results. The results of performing 1000 runs can be seen in figure [ref/figure. (describe the figures and explain what they mean)



The experimental setup to generate the training data is shown in figure [reference figure]
The MATLAB co-simulation is run inside a sub galaxy as shown in 
(describe the experimental setup from VPI here)

The simulation was run in batches of 500 runs, an example of the output from one of these sets of runs can be seen in figure [reference figure]


